\documentclass[10pt]{scrartcl}

\usepackage[utf8]{inputenc}
\usepackage{tabularx}
\usepackage{longtable}
\usepackage[ngerman]{babel}
\usepackage[automark]{scrpage2}
\usepackage{amsmath,amssymb,amstext}
%\usepackage{mathtools}
\usepackage[]{color}
\usepackage[]{enumerate}
\usepackage{graphicx}
\usepackage{lastpage}
\usepackage[perpage,para,symbol*]{footmisc}
\usepackage{listings} 
\usepackage[pdfborder={0 0 0},colorlinks=false]{hyperref}
\usepackage[numbers,square]{natbib}
\usepackage{color}
\usepackage{colortbl}
\usepackage[absolute]{textpos}
\usepackage{float}
%\usepackage[colorinlistoftodos,textsize=small,textwidth=2cm,shadow,bordercolor=black,backgroundcolor={red!100!green!33},linecolor=black]{todonotes}

\lstset{numbers=left, numberstyle=\tiny, numbersep=5pt, breaklines=true, showstringspaces=false} 
\restylefloat{figure}

%changehere
\def\titletext{Bericht TT2P 2}
\def\titletextshort{Problem 3}
\author{Steffen Brauer, André Harms,\\ Florian Johannßen, Jan-Christoph Meier,\\ Florian Ocker, Olaf Potratz,\\ Torben Woggan}

\title{\titletext}

%changehere Datum der Übung
\date{10.06.2012}

\pagestyle{scrheadings}
%changehere
\ihead{TT2, Neitzke}
\ifoot{Generiert am:\\ \today}

\cfoot{Steffen Brauer, André Harms,\\ Florian Johannßen, Jan-Christoph Meier,\\ Florian Ocker, Olaf Potratz,\\ Torben Woggan}


\ohead[]{\titletextshort}
\ofoot[]{{\thepage} / \pageref{LastPage}}

\setlength{\parindent}{0.0in}
\setlength{\parskip}{0.1in}

\begin{document}
\maketitle

\setcounter{tocdepth}{3}
\tableofcontents

%	\listoftables                                 												% 
	\listoffigures  
%	\lstlistoflistings	
\newpage
\section{Frage}

\section{Frage}

\section{Frage}
Wie geht man vor, wenn die Zustandsvariablen, die einen Zustand beschreiben kontinuierlich sind? Wie geht man vor, wenn der Zustandsraum so groß ist, dass der Speicher nicht reicht, um alle V-Werte vzw. Q-Werte abzuspeichern.

Ist die tabellarische Verwaltung der Zustandswerte nicht möglich, da es zuviele Zustände gibt oder sie kontinuierlich sind, müssen sie generalisiert werden. Zusätzlich muss die V- oder Q-Funktion approximiert werden.




\section{Frage}


\end{document}

